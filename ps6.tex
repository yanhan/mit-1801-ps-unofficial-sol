\documentclass[9pt]{article}

\usepackage{amsmath}
\usepackage{tcolorbox}
% `parskip` removes indentation for all paragraphs: http://tex.stackexchange.com/a/55016
\usepackage{parskip}
% Allows us to color rows / cols of a table.
% See https://texblog.org/2011/04/19/highlight-table-rowscolumns-with-color/
\usepackage{color, colortbl}

\usepackage{hyperref}
\graphicspath{{images/ps6/}}

\leftmargin=0.25in
\oddsidemargin=0.25in
\textwidth=6.0in
\topmargin=-0.25in
\textheight=9.25in

\definecolor{Gray}{gray}{0.9}

\begin{document}

\begin{center}
  \large\textbf{MIT 18.01 Problem Set 6 Unofficial Solutions}
\end{center}

\begin{tcolorbox}
  \textbf{Q1)} Do 7.4/12 and 13.
\end{tcolorbox}

\textbf{Skipped.} We do not have the textbook.


\begin{tcolorbox}
  \textbf{Q2)} The voltage $V$ of the house current is given by\\
  \begin{center}
    $V(t) = Csin(120\pi t)$
  \end{center}
  where $t$ is time, in seconds and $C$ is a constant amplitude. The square root of the average value of $V^2$ over one period of $V(t)$ (or cycle) is called the \emph{root-mean-square} voltage, abbreviated RMS. This is what the voltage meter on a house records. For house current, find the RMS in terms of the constant $C$. (The peak voltage delivered to the house is $\pm C$. The units of $V^2$ are square volts; when we take the square root again after averaging, the units become volts again.)
\end{tcolorbox}

Every cycle of the $sin$ function corresponds to $2 \pi$. This happens every $t = \frac{2 \pi}{120 \pi} = \frac{1}{60}$ seconds.

Since $V(t) = Csin(120\pi t)$, $V^2(t) = C^2 sin^2(120\pi t)$. The average value of $V^2$ over 1 cycle of $V(t)$ is:

\begin{align*}
  \frac{1}{\frac{1}{60} - 0} \int_0^{\frac{1}{60}} C^2 sin^2(120\pi t) dt &= 60C^2 \int_0^{\frac{1}{60}} sin^2(120\pi t) dt \\
  &= 60C^2(\frac{1}{2}t - \frac{1}{240\pi} sin(120\pi t) cos(120\pi t)) \bigg]_0^{1/60} \\
  &= 60C^2(\frac{1}{2} \cdot \frac{1}{60} - \frac{1}{240\pi} sin(2\pi) cos(2\pi) - (\frac{1}{2} \cdot 0 - \frac{1}{240\pi} sin(0) cos(0))) \\
  &= 60C^2 \cdot \frac{1}{120} \\
  &= \frac{C^2}{2}
\end{align*}

Then RMS $= \sqrt{\frac{C^2}{2}} = \frac{C}{\sqrt{2}} = \frac{\sqrt{2}C}{2}$


\begin{tcolorbox}
  \textbf{Q3a)} What is the probability that $x^2 < y$ if $(x, y)$ is chosen from the unit square $0 \leq x \leq 1$, $0 \leq y \leq 1$ with probability equal to the area.
\end{tcolorbox}

\begin{center}
  \includegraphics[scale=0.4]{q3a.jpg}
\end{center}

\begin{align*}
  P(x^2 < y) &= \frac{\int_0^1 1 dx - \int_0^1 x^2 dx}{\int_0^1 1 dx} \\
  &= \frac{x \bigg]_0^1 - \frac{x^3}{3}\bigg]_0^1}{x \bigg]_0^1} \\
  &= \frac{1 - \frac{1}{3}}{1} \\
  &= \frac{2}{3}
\end{align*}


\begin{tcolorbox}
  \textbf{Q3b)} What is the probability that $x^2 < y$ if $(x, y)$ is chosen from the square $0 \leq x \leq 2$, $0 \leq y \leq 2$ with probability \textbf{proportional} to the area. (Probability = Part/Whole).
\end{tcolorbox}

\begin{center}
  \includegraphics[scale=0.5]{q3b.jpg}
\end{center}

For the line $y = x^2$ $y = 2$, $x = \sqrt{2}$. Once $x \ge \sqrt{2}$, $y > 2$. Hence $x \leq \sqrt{2}$.

\begin{align*}
  P(x^2 < y) &= \frac{\int_0^2 2 dx - \int_0^{\sqrt{2}} x^2 dx - \int_{\sqrt{2}}^2 2 dx}{\int_0^2 2 dx} \\
  &= \frac{2x \bigg]_0^2 - \frac{x^3}{3} \bigg]_0^{\sqrt{2}} - 2x \bigg]_{\sqrt{2}}^2}{2x \bigg]_0^2} \\
  &= \frac{4 - \frac{(\sqrt{2})^3}{3} - (4 - 2\sqrt{2})}{4} \\
  &= \frac{4 - \frac{2\sqrt{2}}{3} - 4 + 2\sqrt{2}}{4} \\
  &= \frac{2\sqrt{2} - \frac{2\sqrt{2}}{3}}{4} \\
  &= \frac{\frac{6 \sqrt{2} - 2\sqrt{2}}{3}}{4} \\
  &= \frac{\frac{4 \sqrt{2}}{3}}{4} \\
  &= \frac{\sqrt{2}}{3}
\end{align*}


\begin{tcolorbox}
  \textbf{Q3c)} Evaluate \\
  \begin{center}
    $W = \int_0^{\infty} e^{-at} dt = \lim\limits_{N \rightarrow \infty} \int_0^N e^{-at} dt$
  \end{center}
  This is known as an improper integral because it represents the area of an unbounded region. We are using the letter $W$ to signify "whole". \\
  The probability that a radioactive particle will decay some time in the interval $0 \leq t \leq T$ is \\
  \begin{center}
    $P([0, T]) = \frac{\text{PART}}{\text{WHOLE}} = \frac{1}{W} \int_0^T e^{-at} dt$
  \end{center}
  \begin{center}
  \end{center}
  Note that $P([0, \infty)) = 1 = 100\%$
\end{tcolorbox}

\begin{align*}
  \int_0^{\infty} e^{-at} dt &= -\frac{1}{a} e^{-at} \bigg]_0^{\infty} \\
  &= -\frac{1}{a} (e^{-a \cdot \infty} - e^{-a \cdot 0}) \\
  &= -\frac{1}{a} (\frac{1}{e^{a \cdot \infty}} - \frac{1}{e^{a \cdot 0}}) \\
  &= -\frac{1}{a} (0 - \frac{1}{e^0}) \\
  &= -\frac{1}{a} (-1) \\
  &= \frac{1}{a}
\end{align*}

\end{document}
