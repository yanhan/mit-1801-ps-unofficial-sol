\documentclass[9pt]{article}

\usepackage{amsmath}
\usepackage{tcolorbox}
% `parskip` removes indentation for all paragraphs: http://tex.stackexchange.com/a/55016
\usepackage{parskip}
% Allows us to color rows / cols of a table.
% See https://texblog.org/2011/04/19/highlight-table-rowscolumns-with-color/
\usepackage{color, colortbl}

\usepackage{hyperref}
\graphicspath{{images/ps8a/}}

\leftmargin=0.25in
\oddsidemargin=0.25in
\textwidth=6.0in
\topmargin=-0.25in
\textheight=9.25in

\definecolor{Gray}{gray}{0.9}

\begin{document}

\begin{center}
  \large\textbf{MIT 18.01 Problem Set 8A Unofficial Solutions}
\end{center}

\begin{tcolorbox}
  \textbf{4E-2} Find the rectangular equation for $x = t + 1/t$ and $y = t - 1/t$ (compute $x^2$ and $y^2$).
\end{tcolorbox}

\begin{align*}
  x^2 &= t^2 + 2 + 1/t^2 \\
  y^2 &= t^2 - 2 + 1/t^2 \\
  y^2 &= x^2 -4 \\
  x^2 &- y^2 = 4
\end{align*}

This is an equation of a hyperbola centred at the origin with width 2.


\begin{tcolorbox}
  \textbf{4E-3} Find the rectangular equation for $x = 1 + sin\ t, y = 4 + cos\ t$
\end{tcolorbox}

\begin{align*}
  x^2 &= 1 + 2sin\ t + sin^2\ t \\
  y^2 &= 16 + 8sin\ t + cos^2\ t \\
  x^2 + y^2 &= 17 + 2sin\ t + 8cos\ t + sin^2\ t + cos^2\ t \\
  &= 18 + 2(sin\ t + 4cos\ t) \\
  &= 2(1 + sin\ t) + 8(4 + cos\ t) - 16 \\
  &= 2x + 8y - 16
\end{align*}

Then

\begin{align*}
  x^2 - 2x + y^2 - 8y + 16 &= 0 \\
  (x-1)^2 + (y-4)^2 - 1 &= 0 \\
  (x-1)^2 + (y-4)^2 &= 1
\end{align*}

This is an equation of a circle with radius 1 centred at $(1, 2)$.


\begin{tcolorbox}
  \textbf{4E-8} At noon, a snail starts at the center of an open clock face. It creeps at a steady rate along the hour hand, reaching the end of the hand at 1:00 PM. The hour hand is 1 meter long. Write parametric equations for the position of the snail at time t, in some reasonable xy-coordinate system.
\end{tcolorbox}

The average velocity of the snail is 1 metre / h. Let the centre of the clock be the origin.

Treat the positive x-axis as 0 radians. When the hour hand is 12pm, the angle with respect to the positive x-axis is $\pi / 2$ radians. When the hour hand is at 1pm, the angle with respect to the positive x-axis is $\pi / 3$ radians.

Hence, the angle traversed by the hour hand from 12pm to 1pm is $\pi / 6$ radians.

\begin{align*}
  x = t cos\ \theta = t\ cos(\frac{\pi}{2} - \frac{\pi}{6}t) \\
  y = t sin\ \theta = t\ sin(\frac{\pi}{2} - \frac{\pi}{6}t)
\end{align*}

\begin{center}
  \includegraphics[scale=0.8]{p1_4e8.jpg}
\end{center}

Verification:

At $t = 0$ (12pm), the snail should be at $(0, 0)$.

\begin{align*}
  x &= 0\ cos(\frac{\pi}{2} - \frac{\pi}{6} \cdot 0) = 0 \\
  y &= 0\ sin(\frac{\pi}{2} - \frac{\pi}{6} \cdot 0) = 0
\end{align*}

At $t = 1$ (1pm), the snail should be at $(\frac{1}{2}, \frac{\sqrt{3}}{2})$ (draw out a right angle triangle with hypothenus 1 and the angle being $\pi / 3$ radians and you will see why these numbers):

\begin{align*}
  x &= 1\ cos(\frac{\pi}{2} - \frac{\pi}{6} \cdot 1) = cos(\frac{\pi}{3}) = \frac{1}{2} \\
  y &= 1\ sin(\frac{\pi}{2} - \frac{\pi}{6} \cdot 1) = sin(\frac{\pi}{3}) = \frac{\sqrt{3}}{2}
\end{align*}


\begin{tcolorbox}
  \textbf{4F-1d} Find the arclength of $y = (1/3)(2 + x^2)^{3/2},\ 1 \leq x \leq 2$.
\end{tcolorbox}

\begin{align*}
  \frac{dy}{dx} &= \frac{1}{2}(2x)(2 + x^2)^{1/2} = x(2 + x^2)^{1/2} \\
  \sqrt{1 + (\frac{dy}{dx})^2} &= \sqrt{1 + (x(2 + x^2)^{1/2})^2} \\
  &= \sqrt{1 + x^2(2 + x^2)} \\
  &= \sqrt{1 + 2x^2 + x^4} \\
  &= \sqrt{(x^2 + 1)^2} \\
  &= x^2 + 1 \\
\\
  ds &= x^2 + 1\ dx
\end{align*}

Arc length:

\begin{align*}
  \int_{1}^{2} x^2 + 1\ dx &= \frac{x^3}{3} + x \bigg]_{1}^{2} \\
  &= \frac{2^3}{3} + 2 - (\frac{1}{3} + 1) \\
  &= \frac{8}{3} + 2 - \frac{1}{3} - 1 \\
  &= \frac{10}{3}
\end{align*}


\begin{tcolorbox}
  \textbf{4F-4} Find the length of the curve $x = t^2$, $y = t^3$ for $0 \leq t \leq 2$.
\end{tcolorbox}

\begin{align*}
  \frac{dx}{dt} &= 2t \\
  \frac{dy}{dt} &= 3t^2 \\
  ds &= \sqrt{(\frac{dx}{dt})^2 + (\frac{dy}{dt})^2} \\
  &= \sqrt{(2t)^2 + (3t^2)^2} \\
  &= \sqrt{4t^2 + 9t^4} \\
  &= t\sqrt{4 + 9t^2} \\
  \int_{s_0}^{s_1} ds &= \int_{0}^{2} t\sqrt{4 + 9t^2} dt \\
  &= \frac{2/3 \cdot (4 + 9t^2)^{3/2}}{18} \bigg]_{0}^{2} \\
  &= \frac{1}{27} (4 + 9t^2)^{3/2} \bigg]_{0}^{2} \\
  &= \frac{1}{27} ((4 + 9(2)^2)^{3/2} - (4 + 9(0)^2)^{3/2}) \\
  &= \frac{1}{27} ((4 + 36)^{3/2} - 4^{3/2}) \\
  &= \frac{1}{27} (40^{3/2} - 8) \\
\end{align*}


\begin{tcolorbox}
  \textbf{4F-5} Find an integral for the length of the curve given parametrically in Exercise 4E-2 for $1 \leq t \leq 2$. Simplify the integrand as much as possible but do not evaluate.
\end{tcolorbox}

What is given in 4E-2: $x = t + 1 / t$, $y = t - 1 / t$.

\begin{align*}
  \frac{dx}{dt} &= 1 - \frac{1}{t^2} \\
  \frac{dy}{dt} &= 1 + \frac{1}{t^2} \\
  ds &= \sqrt{(\frac{dx}{dt})^2 + (\frac{dy}{dt})^2} dt \\
  &= \sqrt{(1 - \frac{1}{t^2})^2 + (1 + \frac{1}{t^2})^2} dt \\
  &= \sqrt{1 - \frac{2}{t^2} + \frac{1}{t^4} + 1 + \frac{2}{t^2} + \frac{1}{t^4}} dt \\
  &= \sqrt{2 + \frac{2}{t^4}} dt \\
  &= \sqrt{\frac{2t^4 + 2}{t^4}} dt \\
  \int_{s_0}^{s_1} ds &= \int_{1}^{2} \sqrt{\frac{2t^4 + 2}{t^4}} dt
\end{align*}


\begin{tcolorbox}
  \textbf{4F-8)} Find the length of the curve $x = e^t cos\ t$, $y = e^t sin\ t$ for $0 \leq t \leq 10$.
\end{tcolorbox}

\begin{align*}
  x^2 + y^2 &= e^{2t} cos^2\ t + e^{2t} sin^2\ t = e^{2t} \\
  \frac{dx}{dt} &= e^t cos\ t - e^t sin\ t \\
  \frac{dy}{dt} &= e^t sin\ t + e^t cos\ t \\
\end{align*}

\begin{align*}
  (\frac{dx}{dt})^2 + (\frac{dy}{dt})^2 &= e^{2t} cos^2\ t - 2e^{2t} cos\ t\ sin\ t + e^{2t}sin^2\ t + e^{2t}sin^2\ t + 2e^{2t}sin\ t\ cos\ t + e^{2t}cos^2\ t \\
  &= e^{2t} cos^2\ t + e^{2t}sin^2\ t + e^{2t}sin^2\ t + e^{2t}cos^2\ t \\
  &= 2 e^{2t}
\end{align*}

\begin{align*}
  ds &= \sqrt{(\frac{dx}{dt})^2 + (\frac{dy}{dt})^2}\ dt \\
  &= \sqrt{2e^{2t}}\ dt \\
  &= \sqrt{2}\ e^{t}\ dt \\
  \int_{s_0}^{s_1} ds &= \int_{0}^{10} \sqrt{2}\ e^t\ dt \\
  &= \sqrt{2}\ e^t\ \bigg]_{0}^{10} \\
  &= \sqrt{2}\ (e^{10} - e^{0}) \\
  &= \sqrt{2}\ (e^{10} - 1)
\end{align*}


\begin{tcolorbox}
  \textbf{4G-2} Find the area of the segment of $y = 1 - 2x$ in the first quadrant revolved around the x-axis.
\end{tcolorbox}

\begin{align*}
  ds &= \sqrt{1 + (\frac{dy}{dx})^2} \ dx \\
  &= \sqrt{1 + (-2)^2} \ dx \\
  &= \sqrt{5} \ dx
\end{align*}

Surface area:

\begin{align*}
  \int_{s_0}^{s_1} 2 \pi y \ ds &= \int_{0}^{1/2} 2 \pi (1 - 2x) \sqrt{5} dx \\
  &= 2 \sqrt{5} \pi \int_{0}^{1/2} 1 - 2x \ dx \\
  &= 2 \sqrt{5} \pi (x - x^2) \bigg]_{0}^{1/2} \\
  &= 2 \sqrt{5} \pi (\frac{1}{2} - \frac{1}{4})) \\
  &= \frac{\sqrt{5}}{2} \pi
\end{align*}


\begin{tcolorbox}
  \textbf{4G-5} Find the area of $y = x^2$, $0 \leq x \leq 4$ revolved around the y-axis.
\end{tcolorbox}

\begin{align*}
  ds &= \sqrt{1 + (\frac{dy}{dx})^2} \ dx \\
  &= \sqrt{1 + (\frac{1}{2}y^{-\frac{1}{2})^2}} \ dy \\
  &= \sqrt{1 + \frac{1}{4y}} \ dy
\end{align*}

Surface area:

\begin{align*}
  \int_{s_0}^{s_1} 2 \pi x \ ds &= \int_{0}^{16} 2 \pi x \sqrt{1 + \frac{1}{4y}} \ dy \\
  &= 2 \pi \int_{0}^{16} \sqrt{y} \sqrt{1 + \frac{1}{4y}} \ dy \\
  &= 2 \pi \int_{0}^{16} \sqrt{y + \frac{1}{4}} \ dy \\
  &= 2 \pi \frac{2}{3} (y + \frac{1}{4})^{3/2} \bigg]_{0}^{16} \\
  &= \frac{4 \pi}{3} (y + \frac{1}{4})^{3/2} \bigg]_{0}^{16} \\
  &= \frac{4 \pi}{3} ((\frac{65}{4})^{3/2} - \frac{1}{8})
\end{align*}


\begin{tcolorbox}
  \textbf{4H-1b} Give the polar coordinates for the rectangular coordinate $(-2, 0)$
\end{tcolorbox}

$r = 2$, $\theta = \pi$

Verify:

\begin{align*}
  x &= r\ cos \theta = 2 cos\ \pi = 2(-1) = 2 \\
  y &= r\ sin \theta = 2 sin\ \pi = 0
\end{align*}


\begin{tcolorbox}
  \textbf{4H-1f} Give the polar coordinates for the rectangular coordinate $(0, -2)$
\end{tcolorbox}

$r = 2$, $\theta = \frac{3\pi}{2}$

Verify:

\begin{align*}
  x &= r\ cos \theta = 2 cos\ \frac{3\pi}{2} = 0 \\
  y &= r\ sin \theta = 2 sin\ \frac{3\pi}{2} = 2(-1) = -2 \\
\end{align*}

Alternatively, $r = 2$, $\theta = -\frac{\pi}{2}$


\begin{tcolorbox}
  \textbf{4H-1g} Give the polar coordinates for the rectangular coordinate $(\sqrt{3}, -1)$
\end{tcolorbox}

$r = \sqrt{(\sqrt{3})^2 + (-1)^2} = \sqrt{3 + 1} = 2$

We know that $cos\ \theta = \frac{\sqrt{3}}{2}$ and $sin\ \theta = \frac{1}{2}$, so $\theta = \frac{\pi}{6}$

But in this case, we are in the 4th quadrant. So $\theta = 2 \pi - \frac{\pi}{6} = \frac{11 \pi}{6}$ or equivalently, $\theta = -\frac{\pi}{6}$.

Verify:

\begin{align*}
  x &= r\ cos \theta = 2 cos\ \frac{11\pi}{6} = 2 cos\ \frac{\pi}{6} = 2 \cdot \frac{\sqrt{3}}{2} = \sqrt{3} \\
  y &= r\ sin \theta = 2 sin\ \frac{11\pi}{6} = -2 sin\ \frac{\pi}{6} = -2 \cdot \frac{1}{2} = -1
\end{align*}


\begin{tcolorbox}
  \textbf{4H-2a} Find using two different methods the equation in polar coordinates for the circle of radius $a$ with center at $(a, 0)$ on the x-axis, as follows: \\
\\
  (i) write its equation in rectangular coordinates, and then change it to polar coordinates (substitute $x = r\ cos\ \theta$ and $y = r\ sin\ \theta$, and then simplify). \\
\\
  (ii) treat it as a locus problem: let $OQ$ be the diameter lying along the x-axis, and $P: (r, \theta)$ a point on the circle; use $\Delta OPQ$ and trigonometry to find the relation connecting $r$ and $\theta$.
\end{tcolorbox}

For part (i)

Equation of the circle is $(x - a)^2 + y^2 = a^2$.

Let $x = r\ cos\ \theta$, $y = r\ sin\ \theta$. Then

\begin{align*}
  (r\ cos\ \theta - a)^2 + (r\ sin\ \theta)^2 &= a^2 \\
  r^2\ cos^2\ \theta - 2ar\ cos\ \theta + a^2 + r^2\ sin^2\ \theta &= a^2 \\
  r^2 - 2ar\ cos\ \theta &= 0 \\
  r &= 2a\ cos\ \theta
\end{align*}

For part (ii)

\begin{center}
  \includegraphics[scale=0.8]{p1_4h2.jpg}
\end{center}

\begin{align*}
  OQ &= 2a \\
  \angle OPQ &= \frac{\pi}{2} \\
  \angle POQ &= \theta \\
  \angle PQO &= \frac{\pi}{2} - \theta
\end{align*}

\begin{align*}
  cos\ \theta &= \frac{r}{2a} \\
  r &= 2a\ cos\ \theta
\end{align*}


\begin{tcolorbox}
  \textbf{4H-3f} For $r = a\ cos(2\theta)$ (4-leaf rose) \\
  (i) give the corresponding equation in rectangular coordinates; \\
  (ii) draw the graph; indicate the direction of increasing $\theta$
\end{tcolorbox}

For part (i)

\begin{align*}
  x = r\ cos\ \theta &= a\ cos(2 \theta)\ cos(\theta) \\
  y = r\ sin\ \theta &= a\ cos(2 \theta)\ sin(\theta) \\
\\
  x &= a\ cos(2 \theta)\ cos(\theta) \\
    &= a(cos^2 \theta - sin^2 \theta)\ cos\ \theta \\
    &= a(cos^3 \theta - cos\ \theta\ sin^2 \theta) \\
    &= a(cos^3 \theta - cos\ \theta\ (1 - cos^2 \theta)) \\
    &= a(cos^3 \theta - cos\ \theta\ + cos^3 \theta) \\
    &= a(2\ cos^3 \theta - cos\ \theta) \\
\\
  y &= a\ cos(2 \theta)\ sin(\theta) \\
  &= a\ (cos^2 \theta - sin^2 \theta)\ sin(\theta) \\
  &= a\ (1 - sin^2 \theta - sin^2 \theta)\ sin(\theta) \\
  &= a\ (1 - 2\ sin^2 \theta)\ sin(\theta) \\
  &= a\ (sin\ \theta - 2\ sin^3 \theta) \\
\\
  x^2 &= a^2 (2\ cos^3 \theta - cos\ \theta)^2 \\
  &= a^2(4\ cos^6 \theta - 4\ cos^4 \theta + cos^2 \theta) \\
\\
  y^2 &= a^2\ (sin\ \theta - 2\ sin^3 \theta)^2 \\
  &= a^2\ (4\ sin^6\ \theta - 4\ sin^4 \theta + sin^2 \theta) \\
\\
  r &= a\ cos(2\theta) = a(cos^2 \theta - sin^2 \theta) \\
  r^3 &= ar^2\ cos(2\theta) \\
  &= a(r^2 cos^2 \theta - r^2 sin^2 \theta) \\
  &= a(x^2 - y^2)
\end{align*}

Using polar coordinates, we know that $r = \sqrt{x^2 + y^2}$. Therefore $r^3 = (\sqrt{x^2 + y^2})^3$.

Equating $r^3 = a(x^2 - y^2)$ and $r^3 = (\sqrt{x^2 + y^2})^3$:

\begin{align*}
  a(x^2 - y^2) &= \sqrt{(x^2 + y^2)}^{3} \\
  (x^2 + y^2)^{3/2} &= a(x^2 - y^2)
\end{align*}

For part (ii)

\begin{center}
  \begin{tabular}{|c|c|}
    \hline
    \rowcolor{Gray}
    $\theta$ & $r = a\ cos(2 \theta)$ \\ \hline
    $0$ & $a\ cos(0) = a$ \\ \hline
    $\pi / 6$ & $a\ cos(\pi / 3) = a / 2$ \\ \hline
    $\pi / 4$ & $a\ cos(\pi / 2) = 0$ \\ \hline
    $\pi / 3$ & $a\ cos(2 \pi / 3) = -a / 3$ \\ \hline
    $\pi / 2$ & $a\ cos(\pi) = -a$ \\ \hline
    $2 \pi / 3$ & $a\ cos(4 \pi / 3) = -a / 2$ \\ \hline
    $3 \pi / 4$ & $a\ cos(3 \pi / 2) = 0$ \\ \hline
    $5 \pi / 6$ & $a\ cos(5 \pi / 3) = a / 2$ \\ \hline
    $\pi$ & $a\ cos(2 \pi) = a$ \\ \hline
    $7 \pi / 6$ & $a\ cos(7 \pi / 3) = a / 2$ \\ \hline
    $5 \pi / 4$ & $a\ cos(5 \pi / 2) = 0$ \\ \hline
    $3 \pi / 2$ & $a\ cos(3 \pi) = -a$ \\ \hline
    $5 \pi / 3$ & $a\ cos(10 \pi / 3) = -a / 2$ \\ \hline
    $7 \pi / 4$ & $a\ cos(7 \pi / 2) = 0$ \\ \hline
    $11 \pi / 6$ & $a\ cos(11 \pi / 3) = a / 2$ \\ \hline
    $2 \pi$ & $a\ cos(4 \pi) = a$ \\ \hline
  \end{tabular}
\end{center}

\begin{center}
  \includegraphics[scale=0.8]{p1_4h3.jpg}
\end{center}


\begin{tcolorbox}
  \textbf{4I-2} Find the area of one leaf of a three-leaf rose $r = a\ cos(3 \theta)$.
\end{tcolorbox}

\begin{center}
  \begin{tabular}{|c|c|}
    \hline
    \rowcolor{Gray}
    $\theta$ & $r = a\ cos(3 \theta)$ \\ \hline
    $0$ & $a\ cos(0) = a$ \\ \hline
    $\pi / 12$ & $a\ cos(\pi / 4) = \sqrt{2} \ a / 2$ \\ \hline
    $\pi / 6$ & $a\ cos(\pi / 2) = 0$ \\ \hline
    $\pi / 3$ & $a\ cos(\pi) = -a$ \\ \hline
    $\pi / 2$ & $a\ cos(3 \pi / 2) = 0$ \\ \hline
    $2 \pi / 3$ & $a\ cos(2 \pi) = a$ \\ \hline
    $3 \pi / 4$ & $a\ cos(9 \pi / 4) = \sqrt{2} \ a / 2$ \\ \hline
    $5 \pi / 6$ & $a\ cos(5 \pi / 2) = 0$ \\ \hline
    $11 \pi / 12$ & $a\ cos(11 \pi / 4) = -\sqrt{2} \ a / 2$ \\ \hline
    $\pi$ & $a\ cos(3 \pi) = -a$ \\ \hline
  \end{tabular}
\end{center}

\begin{center}
  \includegraphics[scale=0.8]{p1_4i2.jpg}
\end{center}

Assume the petals are of equal area. We will find the area of the curve from $\theta = 0$ to $\theta = \pi / 6$ and multiply by 2.

\begin{center}
  \includegraphics[scale=0.8]{p1_4i2-integration.jpg}
\end{center}

\begin{align*}
  2 \int_0^{\pi / 6} \frac{1}{2} r^2 d \theta &= \int_0^{\pi / 6} r^2 d \theta \\
  &= \int_0^{\pi / 6} a^2\ cos^2(3 \theta) d \theta \\
  &= a^2 \int_0^{\pi / 6} cos^2(3 \theta) d \theta \\
  &= a^2 \int_0^{\pi / 6} \frac{1 + cos(6 \theta)}{2} d \theta \\
  &= \frac{a^2}{2} \int_0^{\pi / 6} 1 + cos(6 \theta) d \theta \\
  &= \frac{a^2}{2} (\theta + \frac{sin(6 \theta)}{6}) \bigg]_0^{\pi / 6} \\
  &= \frac{a^2}{2} (\frac{\pi}{6} + \frac{sin(6 \cdot \pi / 6)}{6}) \\
  &= \frac{a^2}{2} (\frac{\pi}{6} + \frac{sin(\pi)}{6}) \\
  &= \frac{a^2 \pi}{12}
\end{align*}


\begin{tcolorbox}
  \textbf{4I-3} Find the area of the region $0 \leq r \leq e^{3 \theta}$ for $0 \leq \theta \leq \pi$
\end{tcolorbox}

Skipped.


\begin{tcolorbox}
  \textbf{Q1a)} Find the algebraic equation in $x$ and $y$ for the curve \\
  \\
  $x = a\ cos^k t, y = a \ sin^k t$ \\
  \\
  Draw the portion of the curve $0 \leq t \leq \pi / 2$ in the three cases $k = 1, k = 2, k = 3$.
\end{tcolorbox}

Raising both $x$ and $y$ to the power $2/k$, we get

$x^{2/k} = a^{2/k} \ cos^2 t$

$y^{2/k} = a^{2/k} \ sin^2 t$

Summing them, we get $x^{2/k} + y^{2/k} = a^{2/k} cos^2 t + a^{2/k} sin^2 t = a^{2/k}$

For $k = 1$, this is $x^2 + y^2 = a^2$

\begin{center}
  \includegraphics[scale=0.8]{1_keq1.jpg}
\end{center}

For $k = 2$, this is $x + y = a$ or equivalently, $y = a -x$.

Assuming $a > 0$, we get:

\begin{center}
  \includegraphics[scale=0.8]{1_keq2.jpg}
\end{center}

For $k = 3$, this is $x^{2/3} + y^{2/3} = a^{2/3}$

\begin{center}
  \includegraphics[scale=0.8]{1_keq3.jpg}
\end{center}


\begin{tcolorbox}
  Q1b) Without calculation, find the arclength in the cases $k = 1$ and $k = 2$.
\end{tcolorbox}

Frankly speaking, we need calculations for these.

Arc length for $k = 1$: $\frac{\pi a}{2}$

Arc length for $k = 2$: $\sqrt{2a^2}$


\begin{tcolorbox}
  Q1c) Find a definite integral formula for the length of the curve for general $k$. Then evaluate the integral in the three cases $k = 1, k = 2$ and $k = 3$. (Your answer in the first two cases should match what you found in part (b), but the calculation takes more time.)
\end{tcolorbox}

\begin{align*}
  x^{2/k} + y^{2/k} &= a^{2/k} \\
  y^{2/k} &= a^{2/k} - x^{2/k} \\
  y &= (a^{2/k} - x^{2/k})^{k/2} \\
  \frac{dy}{dx} &= \frac{k}{2}(a^{2/k} - x^{2/k})^{k/2 - 1}(-\frac{2}{k}x^{(2/k) - 1}) \\
\\
  ds &= \sqrt{1 + (dy/dx)^2} \ dx \\
  &= \sqrt{1 + (\frac{k}{2}(a^{2/k} - x^{2/k})^{k/2 - 1}(-\frac{2}{k}x^{(2/k) - 1}))^2} \ dx \\
  &= \sqrt{1 + (\frac{k^2}{4}(a^{2/k} - x^{2/k})^{k - 2} \ (\frac{4}{k^2}x^{2(2 - k)/k})} \ dx \\
  &= \sqrt{1 + (a^{2/k} - x^{2/k})^{k - 2} \ (x^{2(2 - k)/k})} \ dx \\
  \\
  \int_{s_0}^{s_1} ds &= \int_{0}^{a} \sqrt{1 + (a^{2/k} - x^{2/k})^{k - 2} \ (x^{2(2 - k)/k})} \ dx \\
\end{align*}

For $k = 1$

\begin{align*}
  \int_{0}^{a} \sqrt{1 + (a^2 - x^2)^{-1} \ x^2} \ dx &= \int_{0}^{a} \sqrt{1 + \frac{x^2}{a^2 - x^2}} \ dx \\
  &= \int_{0}^{a} \sqrt{\frac{a^2}{a^2 - x^2}} \ dx \\
  &= a \int_{0}^{a} \sqrt{\frac{1}{a^2 - x^2}} \ dx \\
\end{align*}

Let $x = a\ sin(u)$. Then $dx = a\ cos(u)\ du$. Substitute into above.

\begin{align*}
  a \int_{0}^{\pi / 2} \sqrt{\frac{1}{a^2 - a^2 sin^2(u)}} \ a \ cos(u) \ du &= a^2 \int_{0}^{\pi / 2} \sqrt{\frac{1}{a^2 cos^2(u)}} \ cos(u) \ du \\
  &= a^2 \int_{0}^{\pi / 2} \frac{cos(u)}{a\ cos(u)} \ du \\
  &= a \int_{0}^{\pi / 2} du \\
  &= a\ u \bigg]_0^{\pi / 2} \\
  &= a\pi / 2
\end{align*}

For $k = 2$

\begin{align*}
  \int_{0}^{a} \sqrt{1 + (a^{2/k} - x^{2/k})^{k - 2} \ (x^{2(2 - k)/k})} \ dx &= \int_0^a \sqrt{1 + (a - x)^0 \ x^{2(2 - 2) / 2}} \ dx \\
  &= \int_0^a \sqrt{1 + 1 \cdot \ x^0} \ dx \\
  &= \int_0^a \sqrt{2} \ dx \\
  &= \sqrt{2}\ x \bigg]_0^a \\
  &= \sqrt{2} \ a
\end{align*}

For $k = 3$

\begin{align*}
  \int_{0}^{a} \sqrt{1 + (a^{2/k} - x^{2/k})^{k - 2} \ (x^{2(2 - k)/k})} \ dx &= \int_{0}^{a} \sqrt{1 + (a^{2/3} - x^{2/3})^{3 - 2} \ (x^{2(2 - 3)/3})} \ dx \\
  &= \int_{0}^{a} \sqrt{1 + (a^{2/3} - x^{2/3}) \ (x^{-2/3})} \ dx \\
  &= \int_{0}^{a} \sqrt{1 + a^{2/3}x^{-2/3} - 1} \ dx \\
  &= \int_{0}^{a} \sqrt{a^{2/3}x^{-2/3}} \ dx \\
  &= \int_{0}^{a} a^{1/3}x^{-1/3} \ dx \\
  &= a^{1/3} \int_{0}^{a} x^{-1/3} \ dx \\
  &= a^{1/3} \cdot \frac{3}{2} x^{2/3} \bigg]_{0}^{a} \\
  &= a^{1/3} \cdot \frac{3}{2} a^{2/3} \\
  &= \frac{3}{2} a
\end{align*}

\end{document}
